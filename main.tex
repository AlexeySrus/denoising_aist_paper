% This is samplepaper.tex, a sample chapter demonstrating the
% LLNCS macro package for Springer Computer Science proceedings;
% Version 2.20 of 2017/10/04
%
\documentclass[runningheads]{llncs}
\usepackage[russian]{babel}%
\usepackage{mathtools}

\usepackage{graphicx}
% Used for displaying a sample figure. If possible, figure files should
% be included in EPS format.
%
% If you use the hyperref package, please uncomment the following line
% to display URLs in blue roman font according to Springer's eBook style:
% \renewcommand\UrlFont{\color{blue}\rmfamily}

\begin{document}
%
\title{Unsupervised training denoising networks}
%
%\titlerunning{Abbreviated paper title}
% If the paper title is too long for the running head, you can set
% an abbreviated paper title here
%
\author{Alexey Kovalenko\inst{1}\orcidID{0000-1111-2222-3333} \and
Yana Demyanenko\inst{1}\orcidID{1111-2222-3333-4444}}
%
\authorrunning{A. Kovalenko \and Y. Demyanenko}
% First names are abbreviated in the running head.
% If there are more than two authors, 'et al.' is used.
%
\institute{Southern Federal University, 	Rostov-on-Don, Russia
\email{sfedu email}\\
\url{https://www.sfedu.ru}}
%
\maketitle              % typeset the header of the contribution
%
\begin{abstract}
This work explore approach for image denoising of received by CMOS sensor. Proposed pipeline solves the problem of unsupervised training neural network architectures for image denoising which uses datasets without clean data. This approach bases on theoretical background about image restoration proposed by Nvidia researchers. We implemented custom denoising neural network architectures using specifics of noise distribution. Networks are trained on custom images collection.

\keywords{Image denoising  \and unsupervised learning \and neural networks \and learning image denoising.}
\end{abstract}
%
%
%
\section{Introduction}
Шумоподавление является часто встречаемой задачей в области компьютерного зрения. Любое изображение, полученное с помощью CMOS сенсора будет содержать шум. Данный шум появляется у полезного сигнала из-за погрешностей приёма оптического излучения сенсором. Матрицу полезного сигнала будем обозначать $I$, а компоненту шума $\alpha$, предполагая, что процесс появления шума является абсолютно случайным процессом из распределения $\mathit{P}$. Тогда матрицу итогового изображения можно обозначить формулой~\ref{eq:matrix_def}.

\begin{equation}\label{eq:matrix_def}
\tilde{I}\ =\ I\ +\ \alpha,\ \alpha \sim \mathit{P}
\end{equation}

Так как погрешность приёма оптического сигнала зависит от физического устройства CMOS сенсора, то для каждой модели сенсора будет уникальное распределение $\mathit{P}$, порождающее шумовую компоненту сигнала.

Целью данной работы является построение с помощью нейронной сети приближения отображения $\phi: \mathit{R}^n \longrightarrow \mathit{R}^n$, обладающим следующим свойством:
\begin{equation}\label{eq:main_property}
\forall \tilde{I} \Longrightarrow \phi(\tilde{I}) = I 
\end{equation}
Для построения приближения отображения $\phi$, нейронная сеть $f$ будет обучаться решать следующую задачу оптимизации:
\begin{eqnarray}\label{eq:main_min_task}
\min_{\mathnormal{w}} \Arrowvert f(\tilde{I}, \mathnormal{w}) - I \Arrowvert_{L_2}\mathit{ ,}
\end{eqnarray}
где $\mathnormal{w}$ - параметры сети $f$, $L_2$ - евклидова норма.


\section{Related works}

Весомый вклад в область обучения шумоподавляющих сетей вносит работа исследователей из компании Nvidia с названием Noise2Noise: Learning Image Restoration without Clean Data. Главной идеей данной работы является использование представления шума в виде композиции чистого сигнала и уникального, шума, полученного в разный момент времени, для обучения нейронной сети восстанавливать чистый сигнал изображения. На вход нейронной сети подается зашумленное изображение с компонентой шума $\alpha_1$ и от сети требуется предсказать то же изображение, но с компонентой шума $\alpha_2$. Предполагая, что шум, получен случайным образом, нейронная сеть не способна его предсказать, и поэтому при обучении нейронная сеть стремиться восстанавливать изображение с некоторыми потерями. Недостатком подхода является использование только пары изображений одной сцены при обучении, тем самым возможны большие потери полезного сигнала при работе сети.

Также существует ряд работ для обучения шумоподавляющих сетей на наборах данных, содержащих изображения без компоненты шума. Примером такого подхода является работа инженеров из компании Google с названием Unprocessing Images for Learned Raw Denoising. А данной работе авторы обучат сеть с использованием стандартной функции ошибок $L_1$ тестируют на наборе данных Darmstadt Noise Dataset. 

Наборы данных по типу Darmstadt Noise Dataset содержат пары изображений. Каждая пара состоит из изображение, снятого при правильно подобранных параметров камеры для съемки и изображения, имеющего шумы, возникающие из-за некорректных параметров. Сети, обученные на таких наборах данных не решают задачу подавления шума, возникающего у CMOS сенсора камеры даже в максимально корректно подобранных параметрах съемки.

Таким образом для обучения сети, ориентированной на подавления шума с определенного CMOS сенсора возникает необходимость в сборе данных и разработки метода обучения сети на них.

\section{Dataset}

\subsection{Collecting dataset}

Изображения для проведения исследования были получены с помощью фиксированного устройства, \textit{Apple iPhone X}, имеющего камеру, состоящую из двух сенсоров, с характеристиками, приведенными в следующем источнике (источник).

С данного устройства были сделаны серии RAW изображений семи сцен с различным освещением и цветовым наполнением. Данные серии получены стандартными средствами для съёмки на устройстве. При такой съемке автоматически применяются алгоритмы улучшения кадра на устройстве, в том числе и подавление шума. Данные алгоритмы являются коммерческой тайной компании Apple. Под сценой подразумевается съёмка неизменяемой картины реального мира, получая матрицу~(\ref{eq:matrix_def}). Для этого устройство фиксировалось на штативе в неподвижном состоянии  и запуск процесса съёмки кадра производился  с беспроводного устройства, тем самым получая набор изображений $\{\tilde{I}^q_k\}_{k=1}^{N}$~(\ref{eq:collection}), где $q$ - порядковый номер снимаемой сцены. Для каждой сцены при съемке были зафиксированны значения ISO, фокуса и цветовой температуры. 

Каждая сцена содержит в среднем по $14$ фотографий, суммарное количество кадров из всех наборов составляет $95$ изображений.

Максимальное количество изображений в одной серии ограничено $20$-ю кадрами, так как при продолжении процесса съёмки фотосенсор нагревается и из-за теплового воздействия появляются дополнительные искажения сигнала, из-за которых теряется попиксельное соответствие кадров в серии между собой.

В итоге получается набор данных, состоящий из $125$ изображений, снятых на один сенсор. Визуальное различие шума на изображениях, полученных разными подходами можно рассмотреть на рисунке~\ref{fig:noise_comprasion}.

Также для сравнения результатов были сделаны 6 серий, снятых следующей web камерой в разрешении $1920\times1080$. Снимались видеозаписи короткой длительности и разбивались на кадры. Суммарное количество кадров в данных сериях составило $811$.

\subsection{Analysis dataset}
Во время процесса съемки при нагреве фотосенсора могут происходить незначительные искажения кадра. Или также возможны незаметные смещения устройства при съемке, а также смена внешних условий, таких как освещение или движение объектов в кадре. Для собираемого набора изображений сдвиг более чем на один пиксель между кадрами одной серии уже критичен.

Для анализа качества полученных серий изображений были построены распределения $e^q$~(\ref{eq:series_distribution}) отклонений каждого изображения от усредненного по всем изображениям из серии $\hat{I}^q$~(\ref{eq:mean_image}) по евклидовой метрике.
\begin{eqnarray}\label{eq:mean_image}
\hat{I}^q_{i,j}\ =\ \frac{\sum_{k}\tilde{I}^q_{k\ i,j}}{N}\textit{, где }N\textit{ количество кадров в выборке }q
\end{eqnarray}

\begin{eqnarray}\label{eq:series_distribution}
e^q_k\ =\ \Arrowvert \hat{I}^q - \tilde{I}^q_k \Arrowvert_{L_2}
\end{eqnarray}
Дополнительно производится нормировка выборки по метрике $L_1$:
$$e^q_k\ =\ \frac{e^q_k}{\sum_{i}(e^q_i)}$$

Далее по значениям выборки $e^q$~(\ref{eq:series_distribution}) строятся графики плотности нормального распределения $\mathcal{N}(\mu, \sigma)$ с параметрами:
\begin{eqnarray}\label{eq:mean_and_std}
\mu^q = \mu(e^q),\ \sigma^q = \sigma(e^q)
\end{eqnarray}

Таким образом математическое ожидание $\mu^q$~(\ref{eq:mean_and_std}) показывает насколько усредненное изображение $\hat{I}^q$~(\ref{eq:mean_image}) близко к изображению, теоретически, получаемому из чистого сигнала $I^q$~(\ref{eq:collection}): 
\begin{eqnarray}\label{eq:mean_image_approximation}
\hat{I}^q \xrightarrow[\mu^q \rightarrow 0]{} I^q
\end{eqnarray}

А также, чем ближе к $0$ значение среднеквадратичного отклонения выборки $\sigma^q$~(\ref{eq:mean_and_std}), тем на большем количестве изображений из серии присутствует сигнал $G(t)$~(\ref{eq:signal_decomposition}), зафиксированный в момент времени $t = t_0$, то есть насколько сцена неизменна между кадрами.

Рассматриваемый параметр $\sigma$~(\ref{eq:mean_and_std}) сильно зависит не только от движения объектов на снимаемой сцене, но и от уровня освещения. Сдвиг объектив особо критичен, так как обученная нейросеть на такой выборке будет размывать результирующие изображение.

Также большое значение отклонения $\sigma$~(\ref{eq:mean_and_std}), полученное из-за сильного изменения освещения при съемке также неблагоприятно скажется на качестве обучения нейронной сети, так как нейронная сеть, обученная на таких сериях, не сможет корректно предсказывать цветовую гамму результирующего изображения. Пример зависимости параметра отклонения от уровня освещения снимаемой сцены при естественном освещении изображен на рисунке~\ref{fig:hists_comparision}. Из данного примера  можно заметить, что незначительные отклонения в освещении, различимые по небольшим отличиям в цветовых гистограммах изображений из серии сильно влияют на параметр $\sigma$~(\ref{eq:mean_and_std}), и для данной статичной сцены данный параметр близок к значению, получаемому у сцены с сильным сдвигом~\ref{fig:deviations_comparision}.

\begin{figure}
	\centering
	\includegraphics[width=\textwidth]{img/imgs_8_series_stack_image}
	\caption{Серия изображений с цветовыми гистограммами, параметр $\sigma$ данной серии равен $0.00608$ }
	\label{fig:hists_comparision}
\end{figure}

\begin{figure}
	\centering
	\includegraphics[width=\textwidth]{img/night_condition_series_real_noise_iphone_deviation_comparison}
	\caption{Графики плотностей нормальных распределений для серий, снятых при условиях плохого искусственного освещения с помощью приложения}
	\label{fig:distribuion_real_noise}
\end{figure}


\begin{figure}
	\centering
	\includegraphics[width=\textwidth]{img/good_condition_series_real_noise_iphone_deviation_comparison}
	\caption{Графики плотностей нормальных распределений для серий, снятых при условиях естественного освещения с помощью приложения}
	\label{fig:distribuion_real_noise_good_condition}
\end{figure}

\begin{figure}
	\centering
	\includegraphics[width=\textwidth]{img/series_webcam_deviation_comparison}
	\caption{Графики плотностей нормальных распределений для серий, снятых при разных условиях освещения с помощью web камеры}
	\label{fig:distribuion_webcam}
\end{figure}


При выборе данных для обучения нейронных сетей подавляющих шум на изображении задается условие, что параметр $\sigma$~(\ref{eq:mean_and_std}) не должен превышать значения $0.007$


\subsubsection{Sample Heading (Third Level)} Only two levels of
headings should be numbered. Lower level headings remain unnumbered;
they are formatted as run-in headings.

\paragraph{Sample Heading (Fourth Level)}
The contribution should contain no more than four levels of
headings. Table~\ref{tab1} gives a summary of all heading levels.

\begin{table}
\caption{Table captions should be placed above the
tables.}\label{tab1}
\begin{tabular}{|l|l|l|}
\hline
Heading level &  Example & Font size and style\\
\hline
Title (centered) &  {\Large\bfseries Lecture Notes} & 14 point, bold\\
1st-level heading &  {\large\bfseries 1 Introduction} & 12 point, bold\\
2nd-level heading & {\bfseries 2.1 Printing Area} & 10 point, bold\\
3rd-level heading & {\bfseries Run-in Heading in Bold.} Text follows & 10 point, bold\\
4th-level heading & {\itshape Lowest Level Heading.} Text follows & 10 point, italic\\
\hline
\end{tabular}
\end{table}


\noindent Displayed equations are centered and set on a separate
line.
\begin{equation}
x + y = z
\end{equation}
Please try to avoid rasterized images for line-art diagrams and
schemas. Whenever possible, use vector graphics instead (see
Fig.~\ref{fig1}).

% \begin{figure}
% \includegraphics[width=\textwidth]{fig1.eps}
% \caption{A figure caption is always placed below the illustration.
% Please note that short captions are centered, while long ones are
% justified by the macro package automatically.} \label{fig1}
% \end{figure}

\begin{theorem}
This is a sample theorem. The run-in heading is set in bold, while
the following text appears in italics. Definitions, lemmas,
propositions, and corollaries are styled the same way.
\end{theorem}
%
% the environments 'definition', 'lemma', 'proposition', 'corollary',
% 'remark', and 'example' are defined in the LLNCS documentclass as well.
%
\begin{proof}
Proofs, examples, and remarks have the initial word in italics,
while the following text appears in normal font.
\end{proof}
For citations of references, we prefer the use of square brackets
and consecutive numbers. Citations using labels or the author/year
convention are also acceptable. The following bibliography provides
a sample reference list with entries for journal
articles~\cite{ref_article1}, an LNCS chapter~\cite{ref_lncs1}, a
book~\cite{ref_book1}, proceedings without editors~\cite{ref_proc1},
and a homepage~\cite{ref_url1}. Multiple citations are grouped
\cite{ref_article1,ref_lncs1,ref_book1},
\cite{ref_article1,ref_book1,ref_proc1,ref_url1}.
%
% ---- Bibliography ----
%
% BibTeX users should specify bibliography style 'splncs04'.
% References will then be sorted and formatted in the correct style.
%
% \bibliographystyle{splncs04}
% \bibliography{mybibliography}
%
\begin{thebibliography}{8}
\bibitem{ref_article1}
Author, F.: Article title. Journal \textbf{2}(5), 99--110 (2016)

\bibitem{ref_lncs1}
Author, F., Author, S.: Title of a proceedings paper. In: Editor,
F., Editor, S. (eds.) CONFERENCE 2016, LNCS, vol. 9999, pp. 1--13.
Springer, Heidelberg (2016). \doi{10.10007/1234567890}

\bibitem{ref_book1}
Author, F., Author, S., Author, T.: Book title. 2nd edn. Publisher,
Location (1999)

\bibitem{ref_proc1}
Author, A.-B.: Contribution title. In: 9th International Proceedings
on Proceedings, pp. 1--2. Publisher, Location (2010)

\bibitem{ref_url1}
LNCS Homepage, \url{http://www.springer.com/lncs}. Last accessed 4
Oct 2017
\end{thebibliography}
\end{document}

